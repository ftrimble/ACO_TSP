\documentclass[twocolumn]{article}


\usepackage{graphicx}
\usepackage{epstopdf}
\usepackage{fancyhdr}

\DeclareMathSizes{36}{36}{36}{36}

\title{Massively Parallel Ant Colony Optimization Applied to the Travelling Salesman Problem}
\author{Forest Trimble, Scott Todd\\trimbf@rpi.edu, todds@rpi.edu}

\begin{document}

\maketitle

\pagestyle{fancy}
\fancyhead{}
\fancyfoot{}
\fancyhead[L]{Trimble, Todd}
\fancyhead[C]{ACO on the TSP}
\fancyhead[R]{\thepage}


\section{Team Member Contributions}

Forest Trimble: \\

\noindent Scott Todd: Performed initial research. Setup the base of the code 
structure, including the input file parsing and parameter initialization. 
Prepared input files and testing materials.



\section{Background}

\subsection{The Travelling Salesman Problem}

The travelling salesman problem (TSP) is an extensively-studied NP-hard problem 
in theoretical computer science with varied applications throughout delivery, 
transportation, planning, and logistic operations. In the formulation of the 
problem, a list of cities is given and the distances between each pair of cities
is known. The question is then: what is the shortest possible path from city to
city that visits each city exactly once? In particular, we studied the symmetric
travelling talesman problem, where the distance from any city A to any city B
is the same as the distance from city B to city A. In this case, the problem
can be modeled as an undirected graph, with vertices representing cities and
edges representing paths between cities.\\

Checking each possible solution to the TSP takes on the order of $O(n!)$ time. 
The dynamic programming solution to the TSP is an exact algorithm (it provably 
returns the optimal solution) which operates in $O(n^22^n)$ time using $O(2^n)$
space. For large cases of the TSP, these large runtimes are prohibitively 
expensive even on supercomputer-class machines. Because of this, approximation 
algorithms and heuristics have been formulated that are able to quickly 
approach the optimal solution, some provably within a certain threshold or with 
a high probability of being particularly close to the optimal solution.\\



\subsection{The Ant Colony Optimization Algorithm}

The Ant Colony Optimization algorithm (ACO) for the travelling salesman problem 
is one such approximation algorithm which lends itself well to parallel 
computation. It was first proposed by Marco Doringo's PhD thesis in 1992. The 
inspiration for this technique comes from the natural world, where ants in a
colony wander seemingly aimlessly until they come across food, at which point
they leave a trail of pheromones for other ants to detect and follow. Pheromones
evaporate over time, so shorter paths accumulate pheromones in a higher density 
more reliably than longer paths. An emergent property of this behavior is that 
efficient paths to food sources will become apparent as more ants wander and 
follow these trails over time.\\

Just as these ants are able to find efficient routes to their food sources by
utilizing this emergent behavior, computers are able to find short paths through
graphs for the TSP by simulating ants and their pheromone trails. Simulated ants
begin at random vertices in the graph of the particular TSP problem, then find a
path through the graph. The simulated ants decide which edge to follow from a
given vertex by using knowledge of the edge weights (distances) and pheromone
levels along each edge. Ants deposit pheromones while walking the graph. \\


\section{Implementation}

\subsection{Algorithm Details}

Each ant $k$ calculates the probability it will move along a given edge $ij$ 
from vertex $i$ to vertex $j$ with the formula:

{\large\begin{equation}p^k_{ij}=\frac{(\tau^\alpha_{ij})(\eta^\beta_{ij})}
{\sum(\tau^\alpha_{{_N}j})(\eta^\beta_{{_N}j})}\end{equation}} 

Where $\tau_{ij}$ is the amount of pheromone on edge $ij$, $\eta_{ij}$ is the
inverse of the length of edge $ij$, and both {$\alpha$} and {$\beta$} are 
heuristic parameters. \\

The amount of pheromone on a given edge changes according to the formula:

{\large\begin{equation}\tau^k_{ij}=(1-\rho)\tau^k_{ij}+\Delta\tau^k_{ij}
\end{equation}} 

Which means that the base amount will decay by a factor of $\rho$ and each ant
that uses that edge will deposit some $\Delta\tau^k_{ij}$ amount of pheromone on
that edge. In many implementations of this algorithm, the $\Delta\tau^k_{ij}$
value is proportional to the path length travelled by ant $k$. \\

An effective implementation of the ACO for TSP algorithm must balance the 
tendancy for ants to follow efficient paths with the desire to discover new,
perhaps more efficient paths. If the simulated ants follow the pheromone trails
too closely, they will quickly get caught in local minimums. The heuristic 
parameters can be chosen to help avoid this. In the 2011 paper titled "An Ant
Colony Optimization Algorithm for Solving Traveling Salesman Problem", the 
researchers experimented with updating the heuristic parameters as the algorithm
executed. \\

\subsection{Parallelizing the Algorithm}

To parallelize this algorithm, we can have multiple ants complete walks of the
graph in parallel then update the pheromone graph with the combined results of
each ant before starting the next iteration.  \\







\section{Related Articles}

blah...\\


\section{Performance Results}

We performed a strong scaling study, where the problem size remained constant
while the processor count increased.\\


\section{Analysis of Performance Results}

big words go here..\\


\section{Summary and Future Work}

optimistic words and lofty goals go here...\\




\end{document}
